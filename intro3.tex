\documentclass[12pt, oneside]{report}

\usepackage[left=2.5cm, top=2.5cm, bottom=2.5cm, right=2.5cm]{geometry}

\usepackage[utf8]{inputenc}
\usepackage[T1]{polski}
\usepackage[polish]{babel}
\usepackage{hyperref}
\usepackage[table]{xcolor}

\begin{document}  
\thispagestyle{empty}
\begin{titlepage}
    \begin{center}

           \Large
	\textbf{Uniwersytet Jagielloński w Krakowie}\vspace{0.2cm}\\ Wydział Fizyki, Astronomii i Informatyki Stosowanej
               \vspace*{1cm}
               
         \vspace{3cm}
         \Large
          \textbf{Paweł Łabno}\\\vspace{0.5cm}
         \normalsize Nr albumu: 1138170\\
             \vspace{2cm}
        \Huge
        \textbf{Implementacja Sztucznej Inteligencji dla gry \textit{Wsiąść do Pociągu}}
      
        \vspace{1.5cm}
        \normalsize
        Praca magisterska\\
        na kierunku Informatyka Stosowana\\ \vspace{0.15cm}
        
        \vfill
        \vspace{2cm}
       \begin{minipage}{1\textwidth}
\begin{flushright}
Praca wykonana pod kierunkiem\\
<tytuł/stopień naukowy Imię Nazwisko>\\
<Instytut/Zakład>
\end{flushright}
\end{minipage}
        
        \vspace{2cm}
        \begin{center}
      Kraków 2018
        \end{center}
    \end{center}
\end{titlepage}

\newpage 
 \thispagestyle{empty}
\vspace{2.5cm}
\begin{flushleft}
\large \textbf{Oświadczenie autora pracy}\vspace{0.6cm}\\
\end{flushleft}

\noindent Świadom odpowiedzialności prawnej oświadczam, że niniejsza praca dyplomowa została napisana przeze mnie samodzielnie i nie zawiera treści uzyskanych w sposób niezgodny z obowiązującymi przepisami.\\

\noindent Oświadczam również, że przedstawiona praca nie była wcześniej przedmiotem procedur związanych z uzyskaniem tytułu zawodowego w wyższej uczelni.
\vspace{2cm}
\begin{center}
\begin{tabular}{lr}
................................~~~~~~~~~~~~~~~~~~~~~~~~~~~~~~~~~~~~~~&
.......................................... \\
{~~~~Kraków, dnia} & {Podpis autora pracy~~~~}
\end{tabular}
\end{center}
\vspace{5cm}
\begin{flushleft}
\large \textbf{Oświadczenie kierującego pracą}
\end{flushleft}

\noindent Potwierdzam, że niniejsza praca została przygotowana pod moim kierunkiem i~kwalifikuje się do przedstawienia jej w postępowaniu o nadanie tytułu zawodowego.
\vspace{2cm}
\begin{center}
\begin{tabular}{lr}
................................~~~~~~~~~~~~~~~~~~~~~~~~~~~~~~~~~~~~~~&
............................................ \\
{~~~~Kraków, dnia} & {Podpis kierującego pracą~~}
\end{tabular}
\end{center}
\vfill
\newpage
\chapter{Wprowadzenie}
\section{Wprowadzenie}
\paragraph{Gra} W literaturze pojawia się wiele definicji dla gry. Można uznać, że jest to czynność o rozrywkowym charakterze, w której uczestniczy jeden lub wielu graczy. Grę można również opisać jako model matematyczny charakteryzujący się określonymi zasadami oraz zbiorem możliwych operacji na tym modelu.
\paragraph{Teoria gier} W przypadku kiedy w rozgrywce uczestniczy więcej niż jeden gracz możemy rozważać zachowania każdego z uczestników. Można założyć, że każdy chce uzyskać jak najlepszy wynik dla siebie.
\paragraph{Klasyfikacja gry}
Gra \textit{Wsiąść do Pociągu} jest grą wieloosobową, niesymetryczną z niepełną informacją. 
\paragraph{Gra z niepełną informacją} 
Określenie gra z niepełną informacją oznacza, że gracze podejmują swoje decyzje nie wiedząc o tym jakie cele mają inni uczestnicy rozgrywki. W grze \textit{Wsiąść do pociągu} polega to m.in. na braku wiedzy jakie bilety (oraz czy zostały ukończone) realizują konkurenci.
\paragraph{Uczenie maszynowe}
Według Donald'a Michie: \\ \textit{System uczący się wykorzystuje zewnętrzne dane empiryczne w celu tworzenia i aktualizacji podstaw dla udoskonalonego działania na podobnych danych w przyszłości oraz wyrażania tych podstaw w zrozumiałej i symbolicznej postaci}
\paragraph{Głębokie sieci neuronowe}
Głębokie sieci neuronowe są szczególnym przypadkiem uczenia maszynowego. Sieć składa się z więcej niż jednej warstwy ukrytej, a reprezentacja wewnętrzna neuronów niekoniecznie jest odworowaniem liniowym.
\section{Zasady gry \textit{Wsiąść do pociągu}}
\subsection{Omówienie celu gry}
Celem rozgrywki jest uzyskanie jak największej liczby punktów. Gracz może zdobyć punkty za zarezerwowanie połączeń \ref{table:points} oraz za ukończenie biletów (korzystając z połączeń jednego gracza można dotrzeć z jednego wskazanego miasta do drugiego). 

\begin{table}[h]
	\begin{center}
		\begin{tabular}{|c|c|} \hline
		\textbf{Długość połączenia} & \textbf{Ilość punktów} \\ \hline
		1 & 1 \\ \hline
		2 & 2 \\ \hline
		3 & 4 \\ \hline
		4 & 7 \\ \hline
		5 & 10 \\ \hline
		6 & 15 \\ \hline
		\end{tabular}	
		\caption{Ilość punktów za zrealizowanie połączenia}
		\label{table:points}
	\end{center}

\end{table}

\subsection{Przygotowanie rozgrywki}
Po rozłożeniu planszy następuje przygotowanie rozgrywki:
\begin{enumerate}
	\item \textbf{Przygotowanie talii}
	\subitem Gracze tasują dostępne karty wagonów oraz biletów. Rozkładają na planszy pierwszych 5 kart wagonów - widocznych typem dla gracza. Jeśli wśród nich są co najmniej 3 lokomotywy - wymienić cały zestaw kart.
	\item \textbf{Losowanie biletów}
	\subitem Każdy z graczy pobiera z talii biletów po trzy karty. Następnie wybiera które z nich zachować a które odłożyć na spód talii. Musi zachować conajmniej dwa bilety. \textit{Przyjmuje się zachowanie jedynie dwóch kart biletów}
	\item \textbf{Losowanie kart wagonów}
	\subitem Każdy z graczy pobiera z talii biletów po cztery karty. Nie ma prawa ich odrzucić lub zamienić.
\end{enumerate}
Gdy zostaną wykonane powyższe kroki uczestnicy gry mogą rozpocząć rozgrywkę. 
\subsection{Możliwości gracza}
W trakcie swojej tury gracz może wybrać jedną z trzech dostępnych akcji:
\begin{enumerate}
	\item \textbf{Zarezerwowanie połączenia} 
	\subitem Każdy z graczy może zarezerwować dowolne połączenie pomiędzy miastami po spełnieniu warunków
	\subitem Gracz musi posiadać odpowiednią ilość kart wagonów odpowiedniego koloru (lokomotywa zastępuje dowolny kolor)
	\subitem Gracz musi posiadać co najmniej tyle wagonów co długość połączenia które chce zarezerwować
	\subitem Gracz nie zarezerwował ścieżki w połączeniu
	\subitem Wszystkie ścieżki w danym połączeniu nie są zajęte
	\subitem Połączenie ma dwie ścieżki, jedna z nich jest zajęta, druga wolna. W rozgrywce uczestniczy co najmniej 4 graczy.
	\subitem \textbf{Decyzja zabroniona} Gracz nie ma wagonów lub kart wagonów potrzebnych do zarezerwowania dowolnego połączenia na mapie.
	\item \textbf{Dobranie wagonów}
	\subitem Gracz może zdecydować o dobraniu do dwóch wagonów w zależności od tego jakie wagony chce dobrać. 
	\subitem Gracz może zabrać jako pierwszy wagon - kartę lokomotywy (Joker) z puli planszy. Nie dobiera wtedy drugiego wagonu.
	\subitem Gracz może zabrać karty z talii lub puli - dwie. Może być jedna karta z talii oraz jedna karta z planszy.
	\subitem \textbf{Decyzja zabroniona} Skończyła się talia kart wagonów, nie ma na planszy kart wagonów a stos kart odrzuconych jest pusty.
	\item \textbf{Dobranie biletów}
	\subitem Gracz w ramach decyzji o dobranie biletów losuje 3 bilety z talii i decyduje, które zachowa, a które odrzuci. Zachować musi co najmniej 1 bilet.
	\subitem \textbf{Decyzja zabroniona} W talii kart biletów nie pozostał żaden bilet.
\end{enumerate}
\subsection{Koniec gry}
Rozgrywka kończy się w momencie, gdy jednemu z graczy pozostają co najwyżej dwa wagoniki. Po tym zdarzeniu każdy z graczy ma jeszcze jeden ruch do wykonaniu. Gdy wszyscy gracze wykonają ruch przechodzi się do fazy podliczenia punktów. \\
Podliczenie przebiega następująco:
\begin{enumerate}
	\item Podliczenie punktów za połączenia (według \ref{table:points} )
	\item Dodanie punktów za każdy zrealizowany bilet (według oznaczenia na bilecie)
	\item Odjęcie punktów za każdy niezrealizowany bilet (według oznaczenia na bilecie)
\end{enumerate}
\textbf{Dla usprawnienia procesu gry w eksperymentach pominięto zasadę bonusowych 10 punktów dla gracza posiadającego najdłuższą nieprzerwaną ścieżkę}
\chapter{Przebieg badania}
\section{Analiza zachowań gracza}
Pierwszym etapem pracy nad sztuczną inteligencją zostało zaanalizowanie zachowań graczy pod punktem przygotowania algorytmu, który wykorzystano do przygotowania zbioru uczącego. Jako rozsądne kierunki analizy przyjąłem:
	\begin{enumerate}
		\item{Analiza rozgrywek z rzeczywistym graczem}
		\item{Analiza rozgrywek z graczem komputerowym (dostępnym z grą w wersji cyfrowej)}
		\item{Przeszukanie sieci internet w tematyce SI}
	\end{enumerate}
W dalszej pracy przyjąłem następujace zachowanie gracza komputerowego:
\begin{enumerate}
	\item Gracz kieruje rozgrywkę dla siebie - zależy mu jedynie na jak największej ilości punktów (w danym momencie oraz ogólnie)
	\item Gracze nie przeszkadzają sobie nawzajem (uczestnicy rozgrywki nie podejmują decyzji mających na celu utrudnienie gry innym uczestnikom, wykluczając sytuację, gdy dla danego gracza decyzja blokująca jest jednocześnie najbardziej korzystną w kontekście zdobyczy punktowej)
\end{enumerate}
\section{Model doświadczenia}
Dla uzyskania jak najbardziej miarodajnych wyników dla każdej przeprowadzanej próby postanowiłem przeprowadzić rozgrywkę 1000 razy w następującej konfiguracji (Tablica \ref{table:gameconfig}):
\begin{table}[h]
	\begin{center}
		\begin{tabular}{| c | c |} \hline
			Liczba graczy & Liczba rozgrywek \\ \hline
			2 & 250 \\ \hline
			3 & 250 \\ \hline
			4 & 250 \\ \hline
			5 & 250 \\ \hline
		\end{tabular}
		\caption{Konfiguracja danych testowych}
		\label{table:gameconfig}
	\end{center}
\end{table}

Następnie w celu wyznaczenia parametrów rozgrywki.  \ref{table:outputparam}

\begin{table}[h]
	\begin{center}
		\begin{tabular}{| c | c |} \hline
			Parametr  & Opis \\ \hline
			MAX & Maksymalna liczba punktów zdobyta przez gracza \\ \hline
			MIN & Najmniejsza liczba punktów zdobyta przez gracza \\ \hline
			AVG & Sredni wynik punktowy \\ \hline
			MED & Mediana punktów \\ \hline
			AVG DONE TCK & Srednia liczba zrealizowanych biletów \\ \hline
			AVG FAIL TCK & Srednia liczba niezrealizowanych biletów \\ \hline
			TURNS & Srednia liczba tur rozgrywki \\ \hline
			110 (\%) & Liczba graczy z co najmniej 110 punktami \\ \hline
			120 (\%) & Liczba graczy z co najmniej 120 punktami \\ \hline
			140 (\%) & Liczba graczy z co najmniej 140 punktami \\ \hline
			FAIL (\%) & Liczba gier w których gracz dokonał ruchu zabronionego \\ \hline
		\end{tabular}
		\caption{Parametry rozgrywki}
		\label{table:outputparam}
	\end{center}
\end{table}


\chapter{Model algorytmiczny}
\section{Algorytm}
\subsection{Decyzja ogólna}
\begin{enumerate}
	\item Gracz nie posiada biletów
	\subitem 1.1 Gracz posiada co najmniej 8 kart wagonów, 8 wagonów, inni gracze mają co najmniej 6 wagonów każdy, pozostały jeszcze bilety w talii - dobranie biletów
	\subitem 1.2 Gracz ma co najmniej 5 kart wagonów jednego koloru oraz może zarezerwować połączenie na planszy - Rezerwacja połączenia
	\subitem 1.3 Gracz może dobrać karty wagonów oraz jest początek rozgrywki - dobranie kart wagonów
	\item Gracz może zarezerwować połączenie na planszy - Rezerwacja połączenia
	\item Gracz może dobrać kartę wagonu - dobranie kart wagonów
	\item Są wolne bilety - dobranie biletów
	\item Gracz jest zmuszony opuścić turę
\end{enumerate}
\subsection{Poddecyzja - bilet}
\begin{enumerate}
	\item \textbf{Przygotowanie} Dla kazdej grupy biletow (możliwego podzbioru pobranych kart) wyznacz koszt (suma brakujących połączeń) oraz możliwą zdobycz punktów zwycięstwa
	\subitem Oznacz grupę biletów, która w sumie spowoduje utratę najmniejszą liczbę punktów - jesli żadnej z nich gracz nie może zrealizować
	\item Z grup, ktore gracz moze ukonczyc wybierz grupe o najniższym koszcie. Jesli dwie lub wiecej grup ma taki sam koszt - wybierz tą, która gwarantuje większą liczbę punktów
	\item Jesli w punkcie drugim nie wybrano żadnej grupy, jako wybraną grupą biletów wybierz grupę oznaczoną w punkcie pierwszym.
\end{enumerate}
\subsection{Poddecyzja - karta wagonów}
\begin{enumerate}
	\item Policz dostepne karty wagonow na planszy (według kolorów) oraz ile kart każdego z kolorów potrzeba na zrealizowanie pozostałych biletów
	\item Wyznacz kolory, których brakuje graczowi a następnie kolor, którego wagonów graczowi brakuje najwięcej.
	\item Jesli gracz ma zrealizowane wszystkie bilety - wez kartę z talii 
	\item Jesli gracz ma zrealizowane wszystkie bilety, a talia sie skonczyla - wez losowa karte z planszy
	\item Jesli graczowi brakuje wiecej niz 3 roznych kolorow kart- wez z talii
	\item Jesli potrzeba mniej niz 4 karty, a pozadany kolor jest na planszy - wez wybrana karte z planszy
	\item Jesli pozostala jedna karta do zrealizowania biletu, gracz nie wybral jeszcze w turze karty oraz jest karta \textit{Lokomotywy} na planszy - wez wybrana karte lokomotywy
	\item Jesli talia nie jest pusta - wez karte z talii
	\item Wez losowa karte z planszy
\end{enumerate}
\subsection{Decyzja - rezerwacja połączenia}
\begin{enumerate}
	\item Oblicz posiadane karty kazdego z kolorow, oraz jakie karty sa potrzebne do rezerwacji polaczen z biletow
	\item Jesli sa karty w zbiorze \textit{Pasujace} oznacz ten zbior jako przetwarzany. W przeciwnym przypadku oznacz zbior kart \textit{mozliwych}
	\item Jako kryterium wyboru polaczenia do rezerwacji wybierz
	\subitem Dla realizacji w przypadku niepustego zbioru \textit{pasujace} wybierz najkrotsze polaczenie
	\subitem W kazdym innym przypadku wybierz polaczenie gwarantujace najwiecej punktow.
	
	\item Okresl pule kolorow do wykorzystania. Jesli jest wiecej niz jeden mozliwy kolor (w przypadku np. teczy), wybierz ten ktorego masz najwiecej kart
	\item Wybierz karty wagonow podanego koloru
\end{enumerate}
\subsection{Przygotowanie tury}
\begin{enumerate}
	\item Okresl pule wszystkich polaczen potrzebnych do realizacji biletow gracza (okreslane jako \textit{cel})
	\item Z wszystkich polaczen na mapie wyznacz te ktore gracz moze zrealizowac w danej turze (okreslane jako \textit{mozliwosci})
	\subitem Dodaj polaczenia zawierajace sie w \textit{celu}
	\subitem Dodaj wszystkie polaczenia o dlugosci co najmniej 5
	\subitem Dodaj wszystkie polaczenia gdy dlugosc zbioru \textit{cel} jest rowna 0
	\item Dla wszystkich polaczen ze zbioru \textit{target} (ktore gracz moze w ogole zarezerwowac)
	\subitem Zawierajace sie w zbiorze \textit{mozliwosci} dodaj do zbioru \textit{pasujace}
	\subitem W przeciwnym przypadku dodaj do zbioru \textit{Brakujace}
	
\end{enumerate}
\subsection{Wyznaczanie tras do realizacji biletów}
Planszę rozgrywki można przedstawić w formie modelu grafu, gdzie połączeniami są krawędzie a wierzchołkami miasta. Do wyznaczenia najbardziej korzystanego połączenia tras wykorzystano Algorytm Bellmana-Forda. Algorytm ten pozwala nam na określenie całej najkrótszej ścieżki rozpoczynając od wierzchołka początkowego. Jako metodę porównywawczą dla kosztu ścieżki uznano:
\begin{enumerate}
	\item Dla różnych kosztów (w wagonów) od węzła początkowego wybieramy ten który jest mniejszy
	\item Dla równego kosztu (w wagonach) od węzła początkowego wybieramy tą ścieżkę dla której ścieżka gwarantuje większą ilość punktów
\end{enumerate}
\section{Wyniki modelu algorytmicznego}
Wyniki uzyskane w doświadczeniu przedstawiono w Tablicy \ref{table:algo_sizeresult}. \textbf{Dane w tabeli są nieaktualne}
\begin{table}[h]	
	\begin{center}
		\begin{tabular}{| c | c | c | c | c | c | c |} \hline
			Liczba graczy & Min pkt & Max pkt & Mediana pkt & zrealizowane bilety & średnio pkt \\ \hline
			2 & 41 & 176 & 104 & 4,29 & 102,49 \\ \hline
			3 & 23 & 151 & 96,5 & 3,53 & 96,34 \\ \hline
			4 & 31 & 173 & 96 & 3,84 &95,99 \\ \hline
			5 & 9 & 176 & 90 & 3,55 & 89,17 \\ \hline
			- & 9 & 176 & 95 & 3,71 &94,56 \\ \hline
		\end{tabular}
		\caption{Wyniki uzależnione od ilości graczy}
		\label{table:algo_sizeresult}
	\end{center}
\end{table}

\section{Wnioski}

\textbf{Wstawić wykres rozkładu punktów}

\chapter{Model sieci neuronowej}
\section{Model danych wejściowych}
\begin{table}[h]
		\begin{tabular}{| c | c | c | c |} \hline
			Nazwa & Typ & Zakres & Opis \\ \hline
			Tura & Int & 0-100 & Aktualna tura gracza \\ \hline
			Karty wagonów & Int & 0-120 & Suma kart wagonów gracza (wszystkich kolorów) \\ \hline
			Wagony na planszy & Int & 0-5 & Ilość wagonów na planszy \\ \hline
			Wagony w talii & Int & 0-120 & Ilość wagonów w talii \\ \hline
			Wagony odrzucone & Int & 0-120 & Ilość odrzuconych wagonów \\ \hline
			Bilety gracza & Int & 0-30 & Ilość biletów posiadanych przez gracza \\ \hline
		\end{tabular}	
		\caption{Dane wejściowe sieci neuronowej}
		\label{table:algo_input}
\end{table}
\section{Model danych wyjściowych}
W eksperymencie wykorzystano klasyfikator, który operuje na pięciu możliwych klasach wyjściowych (do podglądu w \ref{table:algo_classifcator}):
\begin{table}[h]	
	\begin{center}
		\begin{tabular}{| c | c | c |} \hline
			\# & Nazwa & Opis decyzji \\ \hline
			0 & Start & Służy do oznaczenia początku rozgrywki \\ \hline
			1 & Pobranie karty wagonów & Gracz powinien pobrać karty tego typu (akumulacja zasobów) \\ \hline
			2 & Pobranie karty biletów & Gracz powinien pobrać karty tego typu (dobranie celów) \\ \hline
			3 & Zarezerwowanie połączenia & Gracz powinien zabezpieczyć punkty \\ \hline
			4 & Opuszczenie tury & Gracz nie może podjąć żadnej decyzji (sytuacja niemal teoretyczna) \\ \hline
		\end{tabular}
		\caption{Klasy określane przez klasyfikator}
		\label{table:algo_classifcator}
	\end{center}
\end{table}
\section{Struktura sieci}
\section{Proces weryfikacji uczenia}
W trakcie zbierania pomiarów kilkukrotnie zmieniano model danych. Sieć podejmowała często zabronione decyzje spowodowane wpływem niepotrzebnych \textit{featerów}. W związku z tym usunięto dane o poczynaniach innych graczy, a dodano ilość różnych kolorów kart wagonów posiadanych przez gracza oraz największą ilość kart tego samego koloru.
\section{Wyniki} 
\section{Podusmowanie}
\chapter{Dowód dzizałania sieci}
\section{Model eksperymentu}
W celu udowodnienia działania modelu przeprowadziłem eksperyment w formie. Dla każdej epoki uczenia przeprowadziłem serię testów a uśredniony wynik powinien wzrastać wraz z wzrostem iteracji.
\section{Wyniki}
\chapter{Podsumowanie}
\section{Podsumowanie}
\chapter{Słownik}
\paragraph{Gracz}
Uczestnik rozgrywki - sterowany przez sztuczną inteligencję. Przygotowano poniższych uczestników
\subparagraph{Model algorytmiczny} Sztuczna inteligencja działająca w oparciu o algorytm. Wykorzystany do uzyskania danych uczących
\subparagraph{Sieć neuronowa} Sztuczna sieć neuronowa - klasyfikuje rodzaj decyzji jaka ma zostać podjęta przez gracza
\paragraph{Decyzja}
Podejmowana przez uczestnika rozgrywki decyzja (rezerwacja połączenia, pobranie kart wagonów lub pobranie kart biletów)
\subparagraph{Decyzja zabroniona}
Podjęta przez gracza decyzja niemożliwa w danym momencie z punktu widzenia gry. \textit{Przykładowo brakujące zasoby gry}
\paragraph{Bilet}
\label{dictionary:bilet}
Losowane przez uczestnika rozgrywki zadanie do zrealizowania. Określony przez dwa miasta, które gracz musi ze sobą połączyć (za pomocą zarezerwowanych połączeń). W przypadku udanego zrealizowania zadania gracz zdobywa określoną liczbę punktów. W przeciwnym przypadku punkty są odejmowane z zdobytej puli.
\paragraph{Połączenie}
Połączenie dwóch sąsiadujących ze sobą miast. Połączenia charakteryzują się ilością nitek (1 lub dwie), kolorystyką (jeden z 8 kolorów lub połączenie bezbarwne) oraz długością. Można zbudować graf wykorzystując miasta jako węzły oraz połączenia jako krawędzie.
\paragraph{Wagon} Element (w fizycznej wersji gry) służący do znakowania zarezerwowanych połączeń. Jeden wagonik odpowiada jednej karcie wagonów służacej do zarezerwowania połączenia. Gracz na początku posiada 45 wagonów. W momencie, gdy jeden z graczy ma 2 wagony lub mniej gracze wykonują jeszcze po jednym ruchu po czym przechodzą do fazy podliczenia punktów na koniec rozgrywki.
\paragraph{Karta wagonu} Element rozgrywki. W grze występuje jako jeden z 8 zasobów (kolorów) - po 10 sztuk, oraz karta joker (zastępuje dowolną kartę zasobów) - w ilości 12 sztuk. Gracz wykorzystuje karty zasobów do zrealizowania połączenia.
\chapter{Bibliografia}
\begin{enumerate}
	\item{Instrukcja do gry}
	\subitem (Dostęp: ) \\ \url{https://www.wydawnictworebel.pl/repository/files/instrukcje/WdP_USA.pdf}
	\item{Omówienie botów do gry - forum Board Game Geek}
	\subitem (Dostęp: ) \\ 
	\url{https://boardgamegeek.com/thread/1523665/ai-project-solo-multiplayer-games}
	\item{Repozytorium projektu}
	\subitem(Dostęp: ) \\ \url{https://github.com/paqaos/praca-dyplomowa}
\end{enumerate}
\end{document}